\documentclass {article}
\usepackage[usenames]{color}
\usepackage{tikz}
\usepackage{colortbl}
\usepackage{ulem}
\usepackage[pdfborder=0 0 0,colorlinks,linkcolor=black]{hyperref}
\usepackage{verbatim}
\usepackage[utf8]{inputenc}
\usepackage[T1, T2A]{fontenc}
\usepackage[english,russian]{babel}
\usepackage{indentfirst}
\usepackage{listings}
\usepackage{titling}
\usepackage{geometry}
\geometry{pdftex, a4paper}
\geometry{left=2.5cm, right=2.5cm, top = 2cm}
\usepackage{caption}
\DeclareCaptionFont{white}{\color{white}}
\definecolor{mycol}{rgb}{0.98,0.92,0.9}
\definecolor{mycol1}{rgb}{0.5,0.5,0.5}
\frenchspacing   

\DeclareCaptionFormat{listing}{\colorbox{mycol1}{\parbox{\textwidth}{#1#2#3}}}
\captionsetup[lstlisting]{format=listing}
\lstset{%
extendedchars=\true
language=C++,            
basicstyle=\small\ttfamily, 
numbers=left,       
numberstyle=\tiny,       
stepnumber=1,           
numbersep=5pt,            
backgroundcolor=\color{mycol},
showspaces=false,  
showstringspaces=false,  
showtabs=false,     
keywordstyle=\bfseries\color{green!40!black},
identifierstyle=\color{blue},
stringstyle=\color{orange},
frame=single,    
tabsize=2,         
captionpos=t,       
breaklines=true, 
keepspaces = true,
breakatwhitespace=false,
escapeinside={\%*}{*)}   
}
\date{}
\title{{STL. Итераторы}}
\begin{document}
\maketitle
\section{Итератор}
Объект, который синтаксически ведет себя как указатель (определены операторы ++, --, *, -> ).
Универсальный способ перебора элементов контейнеров в STL - перебор с помощью итератора.
Все контейнеры имеют функции, которые возвращают итераторы на первый элемент и на элемент, следующий за последним:
\begin{lstlisting}[caption=Example]
vector< int > v;
vector< int >::iterator begin = v.begin();
vector< int >::iterator end = v.end();
\end{lstlisting}
Для vector и deque реализована арифметика, аналогичная арифметике указателей:
\begin{lstlisting}[caption=Example]
int i = *( v.begin() + 5 );
list< int > l;
list< int >::iterator it = l.begin();
for( ; it != l.end(); ++it )
    cout << *it; Получение объекта по итератору таким способом работает для любого типа контейнеров
list< int >::const_iterator cit = l.begin(); Такой итератор не позволяет менять данные, на которые он указывает
\end{lstlisting}\\
Итераторы позволяют задать позицию в контейнере:
\begin{figure}[h!]
\centering
\includegraphics[width=150px, height=50px]{it.png}
\caption{}
\end{figure}
\begin{lstlisting}[caption=Example]
//Итератор it принадлежит контейнеру v
v.erase( it );// Удаление элемента, на который указывает итератор it
v.insert( it, 5 ); // Вставка элемента 5 на позицию, на которую указывает итератор it
v.insert( it1, 6 ); // Вставка элемента 6 между 2 и 3

*it1 == 3;
*it2 == 7;
*bit == 2; //bit - обратный итератор
// insert( it1, 6 ) и insert( bit, 6 ) дадут одинаковый результат
// Обратный итератор может быть получен следующим образом:
vector< int >::reverse_iterator ri_b = v.rbegin();
vector< int >::reverse_iterator ri_e = v.rend();
// В случае обратного итератора меняется направление прохода по контейнеру:
// оператор ++ передвигает итератор на позицию влево, а оператор -- - на позиуию вправо
erase( it1, it2 ); // Два итератора задают множество элементов, которые будут удалены
insert( it, p, q ); // Итераторы p и q задают последовательность элементов
// другого контейнера, из которого идет вставка элементов
// Итераторы обязательно должны принадлежать одному контейнеру
\end{lstlisting}\\
Возможна инициализация нового контейнера элементами существующего. Последовательность задается с помощью двух итераторов:
\begin{lstlisting}[caption=Example]
vector< int > v; // Контейнер, элементами которого инициализируется новый контейнер
vector< double > w( v.begin(), v.end() );
\end{lstlisting}
\newpage
\section{Источники}
\begin{itemize}
\item \url{http://www.amse.ru/courses/cpp2/2011_02_21.html}
\item \url{https://habr.com/post/122283/}
\item \url{https://habr.com/post/265491/}
\end{itemize}
\newpage
\tableofcontents
\end{document}